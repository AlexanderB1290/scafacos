\chapter{Introduction}
\label{cha:introduction}

\todo[inline]{Fix introduction}

\section{What ScaFaCoS Computes}

\subsection{Fully Non-Periodic Boundary Conditions}
In this part we apply our fast summation algorithm to a system
of $M$ charged particles located at source nodes $\mathbf x_l\in \R^3$ with charge $q_l\in\R$.
We are interested in the evaluation of the electrostatic potential $\phi$ at target node $\mathbf x_j\in \R^3$,
\begin{equation}\label{eq:potential_0dp}
 \phi (\mathbf x_j) = \underset{l \neq j}{\sum_{l=1}^{M}} \dfrac{q_l}{\left\| \mathbf x_j - \mathbf x_l \right\|_2},
 \quad j=1,\hdots,M,
\end{equation}
and the electrostatic fields $\mathbf E$ evaluated at position $\mathbf x_j\in \R^3$,
\begin{equation}\label{eq:fields_0dp}
   \mathbf E(\mathbf x_j)
   = - \underset{l\neq j}{\sum_{l=1}^{M}} q_l \frac{\mathbf x_j - \mathbf x_l}{\|\mathbf x_j - \mathbf x_l \|_2^3},
   \quad j=1,\hdots,M.
\end{equation}
Furthermore we are interested in the computation of the total electrostatic potential energy
\begin{equation*}
  U := \frac{1}{2} \sum\limits_{j = 1}^{M} q_j \phi(\mathbf x_j),
\end{equation*}
which can be evaluated straightforward after the computation of the potentials $\phi(\mathbf x_j), j=1,\hdots,M$.
\todo{Virial}

\subsection{Fully Periodic Systems}
We consider a system of charged particles coupled via the Coulomb potential, a cubic simulation box with
edge length $B$, containing $M$ charged particles, each with a charge $q_l\in\R$, located at $\mathbf x_l\in [0,B)^3$.
Periodic boundary conditions in a system without cut-off is represented by replicating the simulation
box in all directions of space.
The electrostatic potential $\phi$ at $\mathbf y\in [0,B)^3$, can be written as a lattice sum,
see \cite[Chapter 12]{FrSm02} and \cite{Sut06},
\begin{equation}\label{eq:potential_3dp}
  \phi(\mathbf x_j)
  = \sum_{\mathbf r\in \mathbb Z^3} \underset{ l\neq j \,{\rm for}\, \mathbf r= \mathbf 0}{\sum_{l=1}^M}
    \frac{q_l}{\|\mathbf x_j -\mathbf x_l +\mathbf r B\|_2}\,,
    \quad j=1,\hdots,M
\end{equation}
and the electrostatic field $\mathbf E$ evaluated at position $\mathbf x_j\in [0,B)^3$ is given by
\begin{equation}\label{eq:fields_3dp}
  \mathbf E(\mathbf x_j)
  = - \sum_{\mathbf r\in \mathbb Z^3}
    \underset{ l\neq j \,{\rm for}\, \mathbf r= \mathbf 0}{\sum_{l=1}^M}
    q_l \frac{\mathbf x_j -\mathbf x_l +\mathbf r B}{\|\mathbf x_j -\mathbf x_l +\mathbf r B \|_2^3},
    \quad j=1,\hdots,M\, ,
\end{equation}
Furthermore we are interested in the computation of the total electrostatic potential energy
\begin{equation*}
  U := \frac{1}{2} \sum\limits_{j = 1}^{M} q_j \phi(\mathbf x_j),
\end{equation*}
which can be evaluated straightforward after the computation
of the potentials $\phi(\mathbf x_j)$, $j=1,\hdots,M$ like in the non-periodic case.
\todo{Virial}




\section{Acknowledgements}

This is a network project of German research groups working on a
unified parallel library for various methods to solve electrostatic
(and gravitational) problems in large particle simulations. The
project is financed by the German Ministry of Education and Science
(BMBF) under contract number 01 IH 08001 A-D.  Main focus of the
project is to provide methods for electrostatic problems and to
implement efficient parallel methods in order to scale up to thousands
of processors.  \fcs is supported within the period 01/01/2009 -
31/12/2011.


\section{Licensing}

This library is open source software.  It comes with parts that are
distributed under the GNU General Public License (GPL) and parts that
are distributed under the GNU Lesser General Public License (LGPL).
Please refer to the documentation of the individual solver methods for
details.  The toplevel configure script provides a switch
\texttt{--disable-gpl} to disable the build of all methods which allow
only a distribution under the GPL.


\section{Requirements}
\label{sec:requirements}
\index{requirements|main}

The following libraries and tools are required to be able to compile
and use \fcs:

\begin{description}
\item[MPI] \index{MPI} As \fcs is parallel software, you will need a
  working MPI environment that implements at least the MPI standard
  version 1.2.
  \todo{Is MPI 1.2 really enough?}
\item[C99 compiler] \index{C99} The C-code of \fcs uses the C99
  standard, therefore a compiler that is able to compile C99 code must
  be available. We do not know of any recent platform that does not
  provide a C99 compiler.
\item[C++ compiler] \index{C++} Some methods (\eg VMG) require a C++
  compiler.
\item[Fortran 2003 compiler] \index{Fortran 2003} The Fortran code of
  some methods (\eg FMM) does require a Fortran 2003 compiler.
\item[FFTW] \index{FFTW} Some solver methods (\eg P$^3$M and P2NFFT)
  need the FFTW library version 3.3.0 or later
  \footnote{\url{http://www.fftw.org/}} for Fourier transforms.  For
  P2NFFT, the MPI interface of FFTW3 has to be available. You do
  not only need the library itself, but also the header files.
  Depending on the operating system, these may come in separate
  development packages (\eg \texttt{fftw3-dev}).
  \todo{FFTW3 is part of the library}
\end{description}

\section{Feature Matrix}
\label{sec:feature_matrix}
\index{feature matrix|main}

  \begin{table}[htbp]
    \begin{minipage}{\textwidth}
      \centering
      \begin{tabular}{|l||*{10}{c|}}
        \hline
        &
        \rotatebox{90}{\textbf{FMM}} &
        \rotatebox{90}{\textbf{MEMD}} &
        \rotatebox{90}{\textbf{MMM*D}} &
        \rotatebox{90}{\textbf{P2NFFT}} &
        \rotatebox{90}{\textbf{P3M}} &
        \rotatebox{90}{\textbf{PEPC}} &
        \rotatebox{90}{\textbf{PP3MG}} &
        \rotatebox{90}{\textbf{VMG}} &
        \rotatebox{90}{\textbf{Direct}} &
        \rotatebox{90}{\textbf{Ewald}}
        \\
        \hline
        \hline

        \textbf{3D-periodic} &
        + & + & -- & + & + & +                                     & +        & + & + & + \\
        \hspace{2em}\textbf{non-cubic, cuboid} &
        -- & -- &   & + & + & +\footnote{untested}                 & +        & ? & + & + \\
        \hspace{2em}\textbf{non-cubic, non-cuboid} &
        ? & ? &   & -- & ? & +\footnote{untested}                   & ?        & ? & ? & ? \\
        \hspace{2em}\textbf{Virials} &
        ? & -- &   & ?\footnote{diagonal approximation} & ? & --\footnote{implementation planned}   & (scalar) & ? & ? & -- \\
        \hline

        \textbf{Nonperiodic} &
        + & -- & -- & + & -- & +                                   & --        & ? & + & -- \\
        \hspace{2em}\textbf{non-cubic} &
        -- &   &   & + &   & +                                     &          & ? & + &   \\
        \hspace{2em}\textbf{Virials} &
        + &   &   & + &   & --\footnote{implementation planned}    &          & ? & + &   \\
        \hline

        \textbf{Partially periodic} &
        + & -- & + & --\footnote{implementation planned} & -- & +\footnote{untested}                & --        & ? & + & -- \\
        \hline

        \textbf{Tunable accuracy} &
        + & -- & + & + & + & --                                    & --        & -- & -- & + \\
        \hline

        \textbf{Delegate near-field} &
        -- & -- & ? & + & + & --                                   & ?        & ? & -- & -- \\
        \hline

      \end{tabular}
    \end{minipage}
    \caption{Overview of the features of the different solvers.}
  \end{table}
%%% Local Variables: 
%%% mode: latex
%%% TeX-master: ug.tex
%%% End: 

