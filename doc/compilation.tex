\chapter{Compling and installing \fcs}
\label{cha:compiling}

\index{Compilation}
\index{Installation}
\index{SVN}

\todo[inline]{\LaTeX this code}
\begin{verbatim}
0) If you are building the code from the SVN tree, read README-SVN first.

1) Compile:
   a) Run './configure' with suitable options.  See below for common setups.
      You can pass '--enable-fcs-solvers=...' to enable all or only some solvers.
      You can pass '--disable-fcs-SOLVER' to disable a particular solver, e.g.,
      '--disable-fcs-vmg'.
      You can have build trees separate from the source tree, e.g.:
        mkdir ../build-debug
        cd ../build-debug
        ../scafacos_fcs/configure -C CFLAGS=-g FCFLAGS=-g [...]
      See './configure --help' and './configure --help=recursive' for all
      supported options.
      The -C aka. --config-cache option greatly speeds up configuring all the
      subdirectories.
   b) Run 'make'.  Parallel make should work.

2) run test-programs:
   a) Run 'make check'.  This tries to be smart about finding out how to
      start MPI jobs on your system.  If the chosen method is wrong, you
      might have to edit test/defs.in and/or rerun configure with an argument
      of MPIEXEC=... set to the MPI invocation program on your system.

If that doesn't work, then you can alternatively do the following:

   b_1) cd test/c (if you are a c-programmer) or
   b_2) cd test/fortran (if you intend to use fortran)
   c) choose a method you want to run
   d) $mpi_call_on_my_machine ./test_$method ../inp_data/$method/$method_conf.in

3) Sources of the fcs-interface are located in $INSTALL_DIR/src

4) Sources of the ScaFaCos methods can be found in INSTALL_DIR/lib

5) Optionally run 'make install' to install the library and header files.

6) To find out compiler and linker flags to use in packages linking against FCS,
   you can use
      pkg-config --cflags scafacos-fcs
      pkg-config --libs scafacos-fcs

   if you have the pkg-config package installed.

Configure setups for common systems
===================================

Generally, if you set both MPICC and CC, or MPICXX and CXX, or MPIFC and FC,
you may need to ensure that compatible compiler drivers are used.

You may want to add -C to cache configure test results; this speeds up the
recursive configure scripts.

GNU/Linux
---------
It should work to use one of

  ./configure
  ./configure FCFLAGS=-fno-underscoring

depending on which ABI you need to maintain with respect to other Fortran code.
If you are using gfortran, you need version 4.3 or newer for Fortran 2003 support.


BlueGene
--------
The following alternative configurations should work (the --build and --host
arguments ensure configure enables cross-compilation mode):

  ./configure --build=powerpc64-bgp-linux-gnu --host=powerpc-ibm-linux \
     MPICC=mpixlc_r MPIFC=mpixlf2003_r MPICXX=mpixlcxx_r CFLAGS='-qfullpath -O2' \
     FCFLAGS='-qfullpath -qarch=450 -qtune=450 -qxlf2003=nostopexcept -O3 -qipa -qipa=inline=key2addr'

  ./configure --build=powerpc64-bgp-linux-gnu --host=powerpc-ibm-linux \
     MPICC=mpixlc_r MPIFC=mpixlf2003_r MPICXX=mpixlcxx_r \
     CFLAGS='-g -qfullpath -qdbxextra -qcheck' FCFLAGS='-g -qfullpath'

  ./configure --build=powerpc64-bgp-linux-gnu --host=powerpc-ibm-linux \
     MPICC=mpixlc_r MPICXX=mpixlcxx_r MPIFC=mpixlf2003_r \
     CPPFLAGS='-I/bgsys/drivers/ppcfloor/comm/include -I/bgsys/drivers/ppcfloor/arch/include' \
     CFLAGS='-O3 -g -qmaxmem=-1 -qarch=450 -qtune=450' \
     FCFLAGS='-O3 -g -qmaxmem=-1 -I/bgsys/drivers/ppcfloor/include -qarch=450 -qtune=450' \
     LDFLAGS='-L/bgsys/drivers/ppcfloor/lib -L/bgsys/local/lapack/lib -L/bgsys/local/lib' \
     BLAS_LIBS='-lesslbg -lxlf90_r'

Juropa
------

Building with the Intel compiler suite should work:

  ./configure MPICC=mpicc MPICXX=mpicxx MPIFC=mpif90
\end{verbatim}
%%% Local Variables: 
%%% mode: latex
%%% TeX-master: ug.tex
%%% End: 
