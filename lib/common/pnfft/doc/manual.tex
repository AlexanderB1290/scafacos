% \documentclass[a4paper,twoside,11pt,final,openright,headsepline,parskip]{scrreprt}
\documentclass[a4paper,11pt,final,openright,headsepline]{scrreprt}
% parskip: no indentation of paragraphs, german style
%\usepackage{layout} % show layout boarders

\usepackage[english]{babel} % wordbreaks
\usepackage[utf8]{inputenc}
\usepackage[T1]{fontenc}
\usepackage[fixlanguage]{babelbib} % easily change bib language

%% Vector fonts for PDF
% \usepackage{ae} % deprecated package, use lmodern instead
\usepackage{lmodern} %standard latex font

%\usepackage{makeidx} % automatic index generation, required for nomencl.sty
%\usepackage{nomencl} % important symbols in a table at the beginning of document

%% AMSMath-packages
\usepackage{amsmath}
\usepackage{amsthm}
\usepackage{amssymb}
% \usepackage{amsrefs}
% \usepackage{textcmds}
\usepackage{exscale} % Correct font scaling in formulas

\usepackage{subfigure}
\usepackage{graphicx} % Graphics for figures
\usepackage[svgnames, table, hyperref]{xcolor} % 
\usepackage{paralist} % compact itemize, enumerate, ...
% \usepackage{listings} % source code in LaTeX
% \usepackage{multirow} % combine multiple rows in arrays

%% improvements of LaTeX enviroments
\usepackage{scrhack} % avoid warning of scrreprt when loading float package
\usepackage{float}
% \usepackage{verbatim}
% \usepackage{array}

% \usepackage{url} % provides \url command for bibtex
\usepackage{hyperref} % provides \url command for bibtex and links to jump within documents
\hypersetup{plainpages=false, colorlinks, linkcolor=black, citecolor=black, urlcolor=blue,
pdftitle={PNFFT Users Manual},
pdfauthor={Michael Pippig}, pdfstartview={FitBH}}

%% adjust numbering
\numberwithin{figure}{chapter}
\numberwithin{table}{chapter}
\numberwithin{equation}{chapter}

%% activate for algorithms
% \usepackage[chapter]{algorithm}
% \usepackage{algorithmic,algorithmic-fix}
% \floatname{algorithm}{Algorithmus} % use german title for algorithms
% \numberwithin{algorithm}{chapter}

%% activate for compact page layout
\usepackage{geometry}
\geometry{top=30.4mm, left=30.4mm, text={155mm,240mm}, headheight=10mm, headsep=5mm, includemp, marginparwidth=15.4mm}

%% activate for headline with chapter information on every page
\usepackage{scrpage2}
\pagestyle{scrheadings}
\automark[section]{chapter}
\clearscrheadings
\lehead{\pagemark}
\rehead{\leftmark}
\rohead{\pagemark}
\lohead{\rightmark}
\ofoot[]{}
\cfoot[]{}
\ifoot[]{}

%% the quotchap document style redefines the \chapter and \chapter* commands to
%% create fancy chapter head pages with huge chapter numbers (possibly greyed) and
%% provides commands for adding quotations in the upper left corner of these pages.
% \usepackage[grey]{quotchap}


%% very special purpose packages
% \usepackage{faktor} % provides a symbol for factor groups
% \usepackage{slashbox} % diagonaly divide an array field


%% experimental
% \usepackage[color]{showkeys} % show all reference keys
% \definecolor{refkey}{gray}{.75}
% \definecolor{labelkey}{gray}{0.75}
% \usepackage{epstopdf} % include .eps files with pdflatex
% \usepackage{marginnote}
%\usepackage{pgf}
%\usepackage{jkpgf}
%\usepackage{pstricks}
%% make pdf-indexfile for inverse search - needs compatible pdf viewer
% \synctex=1
% \usepackage{pdfsync} % deprecated package, use synctex instead


\hyphenation{equi-spaced non-equi-spaced NFFT FFTW FFT BGFFT nicht-äqui-dis-tant nicht-äqui-dis-tante}

\newtheorem{Satz}{Satz}[chapter] % define Satz enviroment for german docs
\makeatletter
\providecommand{\IfPackageLoaded}[2]{\@ifpackageloaded{#1}{#2}{}}
\providecommand{\IfPackageNotLoaded}[2]{\@ifpackageloaded{#1}{}{#2}}
\providecommand{\IfElsePackageLoaded}[3]{\@ifpackageloaded{#1}{#2}{#3}}
\makeatother


% Shortcuts for math symbols.
\newcommand{\N}{\ensuremath{\mathbb{N}}}
\newcommand{\T}{\ensuremath{\mathbb{T}}}
\renewcommand{\S}{\ensuremath{\mathbb{S}}}
\newcommand{\NZ}{\ensuremath{\mathbb{N}_{0}}}
\newcommand{\Z}{\ensuremath{\mathbb{Z}}}
\newcommand{\R}{\ensuremath{\mathbb{R}}}
\newcommand{\Rp}{\ensuremath{\mathbb{R}_{+}}}
\newcommand{\Rn}{\ensuremath{\mathbb{R}^n}} 
\newcommand{\Rnn}{\ensuremath{\mathbb{R}^{n \times n}}}
\newcommand{\C}{\ensuremath{\mathbb{C}}}
\newcommand{\cO}{\ensuremath{\mathcal{O}}}
\newcommand{\tT}{\ensuremath{\text{\tiny{T}}}}
\newcommand{\ti}{\ensuremath{\text{\scriptsize{i}}}}
\newcommand{\e}{{\ensuremath{\mathrm{e}}}}
\newcommand{\eim}[1]{\ensuremath{\mathrm{e}^{-2\pi{\ti} #1}}}
\newcommand{\eip}[1]{\ensuremath{\mathrm{e}^{+2\pi{\ti} #1}}}
\renewcommand{\mathbf}[1]{\ensuremath{\boldsymbol{#1}}}
\newcommand{\ds}{\displaystyle}
\newcommand{\sinc}{{\rm sinc}}
\newcommand{\dist}{{\rm dist}}
\newcommand{\erf}{{\rm erf}}
\newcommand{\erfc}{{\rm erfc}}
\newcommand{\adj}{{\vdash \hspace*{-1.72mm} \dashv}}
\DeclareMathOperator{\diag}{diag}
\DeclareMathOperator{\dd}{d}
\newcommand{\bend}
{
  \hspace*{0ex} \hfill 
  \hbox{\vrule height 1.5ex \vbox{\hrule width 1.4ex \vskip
  1.4ex\hrule  width 1.4ex} \vrule height 1.5ex}
}
\DeclareMathOperator*{\argmin}{arg\,min}
\DeclareMathOperator*{\argmax}{arg\,max}
\newcommand{\BIGOP}[1]{\mathop{\mathchoice%
{\raise-0.22em\hbox{\huge $#1$}}%
{\raise-0.05em\hbox{\Large $#1$}}{\hbox{\large $#1$}}{#1}}}
\newcommand{\bigtimes}{\BIGOP{\times}}
\def\invisible#1{\textcolor{white}{#1}}
\newcommand{\Vect}[1]{\ensuremath{\mathbf{#1}}}
\newcommand{\Mat}[1]{\ensuremath{\mathbf{#1}}}
\newcommand{\Cal}[1]{\ensuremath{\mathcal{#1}}}
\newcommand{\fft}{\textsf{FFT}}

\newcommand{\ousetarrow}[2]{\overset{\textsf{#1}}{\underset{\textsf{#2}}{\rightarrow}}}
\newcommand{\osetarrow}[1]{\overset{\textsf{#1}}{\rightarrow}}
\newcommand{\usetarrow}[1]{\underset{\textsf{#1}}{\rightarrow}}
\newcommand{\fftarrow}[2]{\overset{\textsf{FFT#1}}{\underset{\textsf{#2}}{\rightarrow}}}
\newcommand{\transposearrow}[1]{\overset{\textsf{T}}{\underset{\textsf{#1}}{\rightarrow}}}

\newcommand{\nfft}{\textsf{NFFT}}
\newcommand{\nfftversion}{3.2.0}
\newcommand{\nfftversionsl}{\textsl{\nfftversion}}

\newcommand{\fftw}{\textsf{FFTW}}
\newcommand{\fftwversion}{3.3.3}
\newcommand{\fftwversionsf}{\textsf{\fftwversion}}
\newcommand{\fftwversionsl}{\textsl{\fftwversion}}

\newcommand{\pfft}{\textsf{PFFT}}
\newcommand{\pfftversion}{1.0.7-alpha}
\newcommand{\pfftversionsf}{\textsf{\pfftversion}}
\newcommand{\pfftversionsl}{\textsl{\pfftversion}}

\newcommand{\pnfft}{\textsf{PNFFT}}
\newcommand{\pnfftversion}{1.0.3-alpha}
\newcommand{\pnfftversionsl}{\textsl{\pnfftversion}}

\newcommand{\ppnfft}{\textsf{P$^2$NFFT}\xspace}
\newcommand{\ppnfftcap}{\texorpdfstring{\ppnfft}{\textsf{P\texttwosuperior NFFT}\xspace}}
\newcommand{\pppm}{\textsf{P$^3$M}\xspace}
\newcommand{\pppmcap}{\texorpdfstring{\pppm}{\textsf{P\textthreesuperior M}\xspace}}

\newcommand{\mybox}{$\Box$}
\newcommand{\myxbox}{{\Large\Crossedbox}}
\newcommand{\mycbox}{{\Large\Checkedbox}}
\newcommand{\myitem}{\item[\mybox]}
\newcommand{\myxitem}{\item[\myxbox]}
\newcommand{\mycitem}{\item[\mycbox]}

\newcommand{\mpipurl}{www.tu-chemnitz.de/~mpip}
\newcommand{\webmpip}{\url{\mpipurl}\xspace}
\newcommand{\websoft}{\url{\mpipurl/software.php}\xspace}


\newcommand{\red}[1]{\textcolor{red}{#1}}

\newcommand{\algname}{Alg.}
\newcommand{\eqname}{Eq.}
\newcommand{\figname}{Fig.}

% \theoremstyle{plain}
%% german
% \newtheorem{Satz}{Satz}[section] % define Satz enviroment for german docs
% \newtheorem{Def}[Satz]{Definition}
% \newtheorem{Lemma}[Satz]{Lemma}
% \newtheorem{Kor}[Satz]{Korollar}
% \newtheorem{Beh}[Satz]{Behauptung}
% \newtheorem{Bsp}[Satz]{Beispiel}
% \newtheorem{Bem}[Satz]{Bemerkung}
% \newenvironment{Beweis}
% {\begin{proof}[Beweis:]}
% {\begin{proof}[\bfseries\upshape Beweis:]}
% {\end{proof}}

%% english ???
%? \newtheorem{Satz}{Satz}[chapter]
%? \newtheorem{Def}[Satz]{Definition}
%? \newtheorem{Lemma}[Satz]{Lemma}
%? \newtheorem{Kor}[Satz]{Korollar}
%? \newtheorem{Beh}[Satz]{Behauptung} 
%? \newtheorem{Bsp}[Satz]{Beispiel} 
%? \newtheorem{Bem}[Satz]{Bemerkung} 
%? \newenvironment{Beweis}
%? % {\begin{proof}[Beweis:]}
%? {\begin{proof}[\bfseries\upshape Beweis:]}
%? {\end{proof}}

% \renewcommand{\qedsymbol}{\#}


%% German commands for algorithms
% \renewcommand{\algorithmicrequire}{\textbf{Eingabe:}}
% \renewcommand{\algorithmicensure}{\textbf{Ausgabe:}}

% activate for listings enviroment
\IfPackageLoaded{listings}
{
  \renewcommand{\ttdefault}{pcr}
  % \renewcommand{\lstlistingname}{\bfseries Quelltext}
  \lstset{language = C,
        morekeywords={ptrdiff_t, fftw_complex, pfft_plan, pfft_gcplan, pfft_timer, pfft_complex, MPI_Comm, size_t, INT},
% 	morekeywords={ptrdiff_t, fftw_complex, pfft_plan, pfft_gcplan, pfft_timer, pfft_complex, MPI_Comm, size_t, INT,
% 	              PX, R, C, X, XM,
% 	              MPI_Init, MPI_Finalize, MPI_Comm_free,
% 	              pfft_init, pfft_destroy_plan, pfft_free,
% 	              pfft_execute,
% 	              pfft_init_input_c2c_3d,
% 	              pfft_plan_dft_3d,
% 	              pfft_alloc_complex,
% 	              pfft_create_procmesh_2d, pfft_create_procmesh,
% 	              pfft_local_size_dft_3d},
        deletekeywords={sizeof,return},
        basicstyle = \ttfamily\small,
	identifierstyle=,
	commentstyle = \slshape\footnotesize,
 	keywordstyle = \bfseries,
	backgroundcolor = \color{LightGray!40},
% 	numbers = left,
	breaklines = true,
  %   	xleftmargin = 3em,
  %   	prebreak = {...},
	breakatwhitespace = true,
	showstringspaces = false,
	escapeinside = {(*@}{@*)},
	moredelim=**[is][\color{red}]{(red@}{@red)},
        moredelim=**[is][\color{green}]{(gre@}{@gre)},
        moredelim=**[is][\color{blue}]{(blu@}{@blu)},
	}
  % 
  % \lstdefinestyle{txt}{language= C, 
  %       basicstyle = \sffamily\small,
  %       breaklines = true,
  %       backgroundcolor = \color{LightGray!0},
  %       numbers=none,
  %        xleftmargin = 0em,
  %       showstringspaces = false}
  % 
  % \lstdefinestyle{head}{
  %       backgroundcolor = \color{LightGray!0},
  %       numbers=none,
  %       belowskip= -2ex}
  % 
%   \newcommand{\code}[1]{\linebreak[2]{\ttfamily #1}}
  \newcommand{\code}[2][\empty]{\ifthenelse{\equal{#1}{\empty}}{\lstinline!#2!}{\lstinline[#1]!#2!}}
}

%% own enviroment for case differentiation
% \newenvironment{FUS}%
% { \renewcommand{\theenumi}{\arabic{enumi}}
%   \renewcommand{\labelenumi}{\itshape Fall \theenumi:}
%   \renewcommand{\theenumii}{\arabic{enumii}}
%   \renewcommand{\labelenumii}{\itshape Fall \theenumi.\theenumii:}
%   \renewcommand{\theenumiii}{\arabic{enumiii}}
%   \renewcommand{\labelenumiii}{\itshape Fall \theenumi.\theenumii.\theenumiii:}
%   \setlength{\labelsep}{1em}
%   \begin{compactenum}
% }
% {\end{compactenum}}
% \settowidth{\labelwidth}{\textit{Fall 44:}}
 % Shortcuts for math symbols


\pagestyle{scrheadings}

\automark[section]{chapter} % auto
\clearscrheadings
\lehead{\pagemark}
\rehead{\leftmark}
\rohead{\pagemark}
\lohead{\rightmark}
\ofoot[]{}
\cfoot[]{}
\ifoot[]{}

% \includeonly{intro} % work on single chapters



\newcommand{\nfftarrow}{\osetarrow{\textsf{NFFT}}}



\begin{document}
  \setcounter{equation}{0}

%   \thispagestyle{empty}
  \begin{titlepage}
    \begin{center}
      \vspace*{3cm}
      {\Huge {\textbf{\textsf{PNFFT Users Manual\\}}}}

      \vspace*{1cm}
      for version \pnfftversion\, \today

      \vspace*{3cm}
      Michael Pippig\\
      Chemnitz University of Technology\\
      Department of Mathematics\\
      09107 Chemnitz, Germany

      \vspace*{3cm}
      Download Parallel Nonequispaced Fast Fourier Transform Software Library at \\
      {\large \url{http://www.tu-chemnitz.de/~mpip/software}}
    \end{center}
    \vfill
    michael.pippig@mathematik.tu-chemnitz.de
  \end{titlepage}

  \tableofcontents


  %%%%%%%%%%%%%%%%%%%%%%%%%%%%%%%%%%%%%%%%%%%%%%%%%%%%%%%%%%%%%%%%%%%%%%%%%%%%%%%
  \chapter{Introduction}\label{kap:intro}
  %%%%%%%%%%%%%%%%%%%%%%%%%%%%%%%%%%%%%%%%%%%%%%%%%%%%%%%%%%%%%%%%%%%%%%%%%%%%%%%
  %%%%%%%%%%%%%%%%%%%%%%%%%%%%%%%%%%%%%%%%%%%%%%%%%%%%%%%%%%%%%%%%%%%%%%%%%%%%%%%%
\chapter{Introduction}\label{chap:intro}
%%%%%%%%%%%%%%%%%%%%%%%%%%%%%%%%%%%%%%%%%%%%%%%%%%%%%%%%%%%%%%%%%%%%%%%%%%%%%%%
This user manual describes the usage of PFFT~\pfftversion~\cite{Pi13,pfft}, a parallel software library for the
computation of equispaced fast Fourier transforms on parallel, distributed memory architectures.
Although the well known FFTW software library~\cite{fftw, FFTW05} already supports parallel FFT
on distributed memory architectures, its one-dimensional data decomposition limits the usage on
massively parallel machines. PFFT can be understood as a generalization of the MPI algorithms
from FFTW to data decompositions of arbitrary dimension.
We refer to the publication~\cite{Pi13} for a closer look on the
different data decompositions and the underlying algorithms of the PFFT library.

The design of PFFT was inspired in many ways by the glorious FFTW library~\cite{fftw,FFTW05} and,
indeed, we tried to implement the PFFT interface as close as possible to the FFTW-MPI interface.
We suppose that the reader is at least familiar with basic usage of FFTW. Since our implementation is based on
MPI, the reader should also know about the basics of distributed memory parallelism and the usage of MPI for parallel
programming.


The PFFT library was designed for distributed memory architectures and was implemented based on the Message Passing Interface (MPI).
We assume that the user of PFFT is at least familiar with the basics of the MPI programming paradigm.
Nevertheless, we explain some of the basic MPI concepts whenever they are important to understand
the design of the PFFT library. We tried to hide most of the parallel coding details from the user,
but of course this is limited by the fact that PFFT itself is a parallel software library.
For more details on the MPI programming language we refer to the MPI Standard~\cite{MPI-2.2}, see also \cite{GrLuTh99} for detailed explanations.
Since the PFFT interface was written as close as possible to the FFTW interface, the user will benefit from






% \section{Why to implement parallel FFT again?}
% As there are several software libraries that deal with the scalability bottleneck, one could ask why we implemented the two-dimensional data decomposition again.
% The answer lies in the usability and flexibility of the available software libraries in comparison to FFTW.
% Our vision is to implement a parallel FFT library, that adepts to the FFTW interface as much as possible
% and preserves most of the nice features that FFTW provides, e.g, high portability, self adapting runtime optimization and in place transforms.
% 
% \section{Comparision to P3DFFT}
% In this section we compare the features and runtimes of PFFT to the P3DFFT software library.
% To get a fair comparison, the P3DFFT feature list has to be up to date. If there is any feature missing, please contact us.


The need of a highly scalable parallel FFT software library can not be over estimated. 

PFFT uses a two-dimensional data decomposition.
PFFT aims to provide a highly scalable FFT implementation

\begin{compactitem}
  \item PFFT is based on FFTW \cite{fftw} and therefore provides most of the features of FFTW
        (arbitrary size DFT in $\Cal{O}(n \log n)$, )
  \item PFFT computes the DFT of complex data
  \item The input data can have arbitrary length.
  \item The number of processes 
\end{compactitem}


\begin{compactitem}
  \item 2d data decomposition if possible (at least 3dFFT)
  \item tries to get same performance as FFTW for the other cases (in fact we only call a wrapper to FFTW)
  \item c2c FFTs
\end{compactitem}

\section{Some Comments on MPI}

Following the MPI standard~\cite{MPI-2.2} we use the term process to denote the smallest
processing unit of a parallel machine.

will adapt to the MPI standard and use the term process to indicate the smallest \\

We stick to the term MPI standard introduces the term process to indicate a \\
The term process abstracts form the widely used terms processors and nodes. \\
Note that the term process abstracts the code development from the real physical architecture. For example we can \\
At first glance, it seems to be artificial to \\
\emph{An MPI program consists of autonomous processes, executing their own code, in an MIMD style.}
( MPI: A Message-Passing Interface Standard, page 27)
\begin{compactitem}
  \item process (identified in most cases with processors/nodes/cores)
  \item plan
\end{compactitem}



\section{Advertisements}
\subsection{Other parallel FFT implementations}
There have been several FFT implementations that aim to circumvent the scalability bottle neck
for at least three dimensional input data by using two-dimensional decomposition approach.
Nevertheless, these implementations are often fitted to special problems and where not published
as a stand alone software library. Remarkable exceptions are the parallel FFT software library by S.~Plimpton~\cite{Pl97,sandiafft},
the P3DFFT software library by D.~Pekurovsky~\cite{Pe12,p3dfft} and the \mbox{2DECOMP\&FFT} software library by N.~Li~\cite{Li2010, 2decompfft}.

\subsection{Parallel nonequispaced FFT}
If your are interested in a parallel implementation of nonequispaced fast Fourier
transforms for distributed memory architectures, you should have a look at our PNFFT software library~\cite{pnfft, PiPo13}.
It is also available at \websoft.

 % ohne Seitenumbrueche
  %%%%%%%%%%%%%%%%%%%%%%%%%%%%%%%%%%%%%%%%%%%%%%%%%%%%%%%%%%%%%%%%%%%%%%%%%%%%%%%
\chapter{Introduction}\label{chap:intro}
%%%%%%%%%%%%%%%%%%%%%%%%%%%%%%%%%%%%%%%%%%%%%%%%%%%%%%%%%%%%%%%%%%%%%%%%%%%%%%%
This user manual describes the usage of PFFT~\pfftversion~\cite{Pi13,pfft}, a parallel software library for the
computation of equispaced fast Fourier transforms on parallel, distributed memory architectures.
Although the well known FFTW software library~\cite{fftw, FFTW05} already supports parallel FFT
on distributed memory architectures, its one-dimensional data decomposition limits the usage on
massively parallel machines. PFFT can be understood as a generalization of the MPI algorithms
from FFTW to data decompositions of arbitrary dimension.
We refer to the publication~\cite{Pi13} for a closer look on the
different data decompositions and the underlying algorithms of the PFFT library.

The design of PFFT was inspired in many ways by the glorious FFTW library~\cite{fftw,FFTW05} and,
indeed, we tried to implement the PFFT interface as close as possible to the FFTW-MPI interface.
We suppose that the reader is at least familiar with basic usage of FFTW. Since our implementation is based on
MPI, the reader should also know about the basics of distributed memory parallelism and the usage of MPI for parallel
programming.


The PFFT library was designed for distributed memory architectures and was implemented based on the Message Passing Interface (MPI).
We assume that the user of PFFT is at least familiar with the basics of the MPI programming paradigm.
Nevertheless, we explain some of the basic MPI concepts whenever they are important to understand
the design of the PFFT library. We tried to hide most of the parallel coding details from the user,
but of course this is limited by the fact that PFFT itself is a parallel software library.
For more details on the MPI programming language we refer to the MPI Standard~\cite{MPI-2.2}, see also \cite{GrLuTh99} for detailed explanations.
Since the PFFT interface was written as close as possible to the FFTW interface, the user will benefit from






% \section{Why to implement parallel FFT again?}
% As there are several software libraries that deal with the scalability bottleneck, one could ask why we implemented the two-dimensional data decomposition again.
% The answer lies in the usability and flexibility of the available software libraries in comparison to FFTW.
% Our vision is to implement a parallel FFT library, that adepts to the FFTW interface as much as possible
% and preserves most of the nice features that FFTW provides, e.g, high portability, self adapting runtime optimization and in place transforms.
% 
% \section{Comparision to P3DFFT}
% In this section we compare the features and runtimes of PFFT to the P3DFFT software library.
% To get a fair comparison, the P3DFFT feature list has to be up to date. If there is any feature missing, please contact us.


The need of a highly scalable parallel FFT software library can not be over estimated. 

PFFT uses a two-dimensional data decomposition.
PFFT aims to provide a highly scalable FFT implementation

\begin{compactitem}
  \item PFFT is based on FFTW \cite{fftw} and therefore provides most of the features of FFTW
        (arbitrary size DFT in $\Cal{O}(n \log n)$, )
  \item PFFT computes the DFT of complex data
  \item The input data can have arbitrary length.
  \item The number of processes 
\end{compactitem}


\begin{compactitem}
  \item 2d data decomposition if possible (at least 3dFFT)
  \item tries to get same performance as FFTW for the other cases (in fact we only call a wrapper to FFTW)
  \item c2c FFTs
\end{compactitem}

\section{Some Comments on MPI}

Following the MPI standard~\cite{MPI-2.2} we use the term process to denote the smallest
processing unit of a parallel machine.

will adapt to the MPI standard and use the term process to indicate the smallest \\

We stick to the term MPI standard introduces the term process to indicate a \\
The term process abstracts form the widely used terms processors and nodes. \\
Note that the term process abstracts the code development from the real physical architecture. For example we can \\
At first glance, it seems to be artificial to \\
\emph{An MPI program consists of autonomous processes, executing their own code, in an MIMD style.}
( MPI: A Message-Passing Interface Standard, page 27)
\begin{compactitem}
  \item process (identified in most cases with processors/nodes/cores)
  \item plan
\end{compactitem}



\section{Advertisements}
\subsection{Other parallel FFT implementations}
There have been several FFT implementations that aim to circumvent the scalability bottle neck
for at least three dimensional input data by using two-dimensional decomposition approach.
Nevertheless, these implementations are often fitted to special problems and where not published
as a stand alone software library. Remarkable exceptions are the parallel FFT software library by S.~Plimpton~\cite{Pl97,sandiafft},
the P3DFFT software library by D.~Pekurovsky~\cite{Pe12,p3dfft} and the \mbox{2DECOMP\&FFT} software library by N.~Li~\cite{Li2010, 2decompfft}.

\subsection{Parallel nonequispaced FFT}
If your are interested in a parallel implementation of nonequispaced fast Fourier
transforms for distributed memory architectures, you should have a look at our PNFFT software library~\cite{pnfft, PiPo13}.
It is also available at \websoft.

 % mit Seitenumbruechen
  \chapter{Fotran Interface}

based on Fortran 90

%------------------------------------------------------------------------------
\subsubsection*{How to Deal with FFT Shifts in Parallel}
%------------------------------------------------------------------------------
PFFT calculates
\begin{equation*}
  h_l = \sum_{k=0}^{N-1} \hat h_k \eim{kl/n},
  \quad l=0,\hdots,L
\end{equation*}
with $n\ge N$ and $n\ge L$.
Step 2 of our PNFFT algorithm requires the shifted index sets
\begin{equation*}
  g_l = \sum_{k=-N/2}^{N/2-1} \hat g_k \eim{kl/n},
  \quad l=-L/2,\hdots,L/2-1
\end{equation*}
A common technique to deal with this problem is to call FFT shift, i.e., set
\begin{equation*}
  \hat h_k
  =
  \begin{cases}
    g_k,     & k=0,\hdots,N/2-1 \\
    g_{k+N}, & k=N/2,\hdots,N-1
  \end{cases}
\end{equation*}
\begin{equation*}
  g_l
  =
  \begin{cases}
    h_{l+N}, & l=-N/2,\hdots,-1 \\
    h_l,     & l=0,\hdots,N/2-1
  \end{cases}
\end{equation*}
However, because of the parallel data decomposition, this involves explicit data communication.
Instead we apply the FFT shifts in frequency domain.
Use the translation property
\begin{equation*}
  \hat h_k = \hat g_k \eip{kL/2/n}
\end{equation*}
and the modulation property
\begin{equation*}
  h_{l-N/2} = \eim{lN/2/n}
\end{equation*}
of the discrete Fourier transform.

User can choose to shift the input (\verb+PNFFT_SHIFT_INPUT+) and/or to shift the output (\verb+PNFFT_SHIFT_OUTPUT+).

One-dimensional example
\begin{align*}
  f(x)
  &=
    \sum_{k=-\frac{N}{2}}^{\frac{N}{2}-1} \hat f_k \eim{kx} \\
  &\approx
    \sum_{l=-\frac{n}{2}}^{\frac{n}{2}-1}\left( \sum_{k=-\frac{N}{2}}^{\frac{N}{2}-1} \frac{\hat f_k}{\varphi_k} \eim{k\frac{l}{n}} \right) \varphi\left(x-\frac{l}{n}\right) \\
  &=
    \sum_{l=-\frac{n}{2}+\frac{n_o}{2}}^{\frac{n}{2}+\frac{n_o}{2}-1} \left(
      \sum_{k=0}^{N-1} \frac{\hat f_{(k-\frac{N}{2})}}{\varphi_{(k-\frac{N}{2})}} \eim{(k-\frac{N}{2})\frac{(l-\frac{n_o}{2})}{n}}
    \right) \varphi\left( \left(x+\frac{n_o}{2n}\right)-\frac{l}{n}\right) \\
  &=
    \sum_{l=-n/2+n_o/2}^{n/2+n_o/2-1} \left[
      \e^{+\pi\ti N\frac{l}{n}} \sum_{k=0}^{N-1} \left(
        \frac{\hat f_{(k-\frac{N}{2})}}{\hat\varphi_{(k-\frac{N}{2})}}
        \e^{+\pi\ti(k-\frac{N}{2})\frac{n_o}{n}}
      \right)
      \eim{k\frac{l}{n}}
    \right] \varphi\left( x-\frac{l-\frac{n_o}{2}}{n} \right)
\end{align*}
\begin{align*}
  \textrm{lo}
  &=
    \frac{\textrm{local\_no\_start} - \textrm{no/2}}{\textrm{n}} \\
  \textrm{up}
  &=
    \frac{\textrm{local\_no\_start} + \textrm{local\_no} - \textrm{no/2}}{\textrm{n}}
  =
    \textrm{lo} + \frac{\textrm{local\_no}}{\textrm{n}}
\end{align*}




%%%%%%%%%%%%%%%%%%%%%%%%%%%%%%%%%%%%%%%%%%%%%%%%%%%%%%%%%%%%%%%%%%%%%%%%%%%%%%%
% \selectbiblanguage{english}
\bibliographystyle{bababbrv-fl}
% \bibliographystyle{abbrv}
\bibliography{bibliography}
%%%%%%%%%%%%%%%%%%%%%%%%%%%%%%%%%%%%%%%%%%%%%%%%%%%%%%%%%%%%%%%%%%%%%%%%%%%%%%%

\end{document}