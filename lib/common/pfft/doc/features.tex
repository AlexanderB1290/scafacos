%%%%%%%%%%%%%%%%%%%%%%%%%%%%%%%%%%%%%%%%%%%%%%%%%%%%%%%%%%%%%%%%%%%%%%%%%%%%%%%
\chapter{Advanced Features}\label{chap:feat}
%%%%%%%%%%%%%%%%%%%%%%%%%%%%%%%%%%%%%%%%%%%%%%%%%%%%%%%%%%%%%%%%%%%%%%%%%%%%%%%


\section{Supported kinds of FFT}

\subsection{c2c-FFT}
FFT of Complex inputs with complex outputs

\subsection{r2c-FFT and c2r-FFT}
FFT of real inputs with Hermitian symmetric complex outputs

\subsection{r2r-FFT}
FFT of real inputs with real outputs



\begin{compactitem}
  \item how to do in place transforms
\end{compactitem}


\section{Reduction of communication by transposed data layouts}
By providing \code{pfft\_flags} the user can choose between three different data layouts of the output
array. Table~\ref{tab:pfft_flags} shows the flags and their corresponding decompositions.
\begin{table}[h]
  \begin{tabular}{lll}
    flag & input layout & output layout \\
    \code{PFFT\_DEFAULT}     & n0/P0 x n1/P1 x n2 & n0/P1 x n1/P0 x n2 \\
    \code{PFFT\_TRANSPOSED}  & n0/P0 x n1/P1 x n2 & n2/P1 x n1/P0 x n0 \\
    \code{PFFT\_P3DFFT}      & n0/P0 x n1/P1 x n2 & n2/P0 x n0/P1 x n1
  \end{tabular}
  \caption{Impact of predefined PFFT flags on the parallel data layout.}
  \label{tab:pfft_flags}
\end{table}


Assume a three dimensional input function \code{fftw\_complex input(ptrdiff\_t k0, ptrdiff\_t k1, ptrdiff\_t k2);}
Data init:
\begin{lstlisting}
int m=0;
for(int k0=0; k0<local_ni[0]; k0++)
  for(int k1=0; k1<local_ni[1]; k1++)
    for(int k2=0; k2<local_ni[2]; k2++)
      in[m++] = input(local_i_start[0]+k0, local_i_start[1]+k1, local_i_start[2]+k2);
\end{lstlisting}


%------------------------------------------------------------------------------
\section{How to Deal with FFT Shifts in Parallel}
%------------------------------------------------------------------------------
A common problem is that the index of the FFT inputs runs between $-N/2,\hdots,N/2-1$, but the FFT library
requires them to run between $0,\hdots,N-1$. With serial program execution one can easily remap the data suitable
for the library, which is also known as \code{fftshift}, e.g., in Matlab.
In contrast, with distributed memory the data movement is more complex and results in a global communication.
Fortunately, this communication can be avoided at the cost of some extra computations, which is often the better
choice on massively parallel, distributed memory architectures.
At the end of the section we present two PFFT library functions that perform the necessary pre- and postprocessing
for shifted input and output index sets.

In a more general setting, we are interested in the computation of FFTs with shifted index sets, i.e., assume $k_s,l_s\in\Z$ and compute
\begin{equation*}
  g_l = \sum_{k=k_s}^{n_i+k_s-1} \hat g_k \eim{kl/n},
  \quad l=l_s,\hdots,n_o+l_s-1\,.
\end{equation*}
Because of the periodicity of the FFT this can be easily performed by \algname~\ref{alg:fftshift_translation}.
\begin{algorithm}\label{alg:fftshift_translation}
  \begin{algorithmic}[1]
    \itemsep=1.1ex
    \State For $k=0,\hdots,n_i-1$ assign $\hat f_k = \hat g_{(k+k_s\mod n_i)}$.
    \State For $l=0,\hdots,n_o-1$ compute $f_l = \sum_{k=0}^{n_i} \hat f_k \eim{kl/n}$ using PFFT.
    \State For $l=0,\hdots,n_o-1$ assign $g_l = f_{(l-l_s\mod n_o)}$.
  \end{algorithmic}
  \caption{Shifted FFT with explicit data movement.}
\end{algorithm}
However, this involves explicit data movement since the sequence of data changes.
For a our parallel data decomposition the change of data layout requires data communication.
A simple index shift results in the computation of
\begin{align*}
  g_{l+l_s}
  &=
    \sum_{k=k_s}^{n_i+k_s-1} \hat g_k \eim{k(l+l_s)/n}
    =
    \sum_{k=0}^{n_i-1} \hat g_{k+k_s} \eim{(k+k_s)(l+l_s)/n} \\
  &=
    \eim{k_sl/n} \sum_{k=0}^{n_i-1} \underbrace{\left(\hat g_{k+k_s}\eim{(k+k_s)l_s/n}\right)}_{=: \hat f_k} \eim{kl/n}
\end{align*}
for all $l=0,\hdots,n_o-1$. The resulting \algname~\ref{alg:fftshift_modulation} preserves the sequence of
data at the price of some extra computation.
\begin{algorithm}\label{alg:fftshift_modulation}
  \begin{algorithmic}[1]
    \itemsep=1.1ex
    \State For $k=0,\hdots,n_i-1$ compute $\hat f_k = \hat g_{k+k_s} \eim{(k+k_s)l_s/n}$.
    \State For $l=0,\hdots,n_o-1$ compute $f_l = \sum_{k=0}^{n_i} \hat f_k \eim{kl/n}$ using PFFT.
    \State For $l=0,\hdots,n_o-1$ compute $g_l = f_l \eim{k_sl/n}$.
  \end{algorithmic}
  \caption{Shifted FFT without explicit data movement.}
\end{algorithm}

The special case $k_s=-\frac{n_i}{2}, l_s=-\frac{n_o}{2}$ corresponds to the shifts the arrays (\textsf{FFTSHIFT})
\begin{algorithm}
  \begin{algorithmic}[1]
    \itemsep=1.1ex
    \State For $k=0,\hdots,n_i-1$ compute $\hat f_k = \hat g_{k-\frac{n_i}{2}} \eim{(k-\frac{n_i}{2})(-\frac{n_o}{2})/n}$.
    \State For $l=0,\hdots,n_o-1$ compute $f_l = \sum_{k=0}^{n_i} \hat f_k \eim{kl/n}$ using PFFT.
    \State For $l=0,\hdots,n_o-1$ compute $g_l = f_l \eim{(-\frac{n_i}{2})l/n}$.
  \end{algorithmic}
\end{algorithm}


In PNFFT: User can choose to shift the input (\verb+PNFFT_SHIFT_INPUT+) and/or to shift the output (\verb+PNFFT_SHIFT_OUTPUT+).





%------------------------------------------------------------------------------
\section{Computation of block distribution}
%------------------------------------------------------------------------------
\subsection{One-dimensional distribution}
%------------------------------------------------------------------------------
Assume $N$ grid points to be distributed on $P$ MPI processes. The block size $B$ is given by
\begin{equation*}
  B = \left\lceil \frac{N}{P} \right\rceil\,.
\end{equation*}
The first $R=\left\lfloor\frac{P}{B}\right\rfloor$ processes get $B$ grid points. The next process gets $N-BR$ grid points and all remaining processes get nothing.


\subsection{Remap three- into two-dimensional distribution}
%------------------------------------------------------------------------------
\begin{equation*}
  \frac{n_0}{P_0} \times \frac{n_1}{P_1} \times \frac{n_2}{Q_0Q_1}
  \to
  \frac{n_0}{P_0} \times \frac{n_1}{P_1} \times \frac{n_2}{Q_0Q_1}
  \to
  \frac{n_0}{P_0Q_0} \times \frac{n_1}{P_1Q_1} \times n_2
\end{equation*}

The algorithm work as follows:
\begin{align*}
  \frac{n_0}{P_0} \times \frac{n_1}{P_1} \times \frac{n_2}{Q_0Q_1}
  &\to
  \frac{n_2}{Q_0Q_1} \times \frac{n_0}{P_0} \times \frac{n_1}{P_1}
  &\to
  \frac{n_2}{Q_0} \times \frac{n_0}{P_0} \times \frac{n_1}{P_1Q_1} \\
  &\to
  n_2 \times \frac{n_0}{P_0Q_0} \times \frac{n_1}{P_1Q_1}
  &\to
  \frac{n_0}{P_0Q_0} \times \frac{n_1}{P_1Q_1} \times n_2
\end{align*}

\subsubsection{Example: $\mathbf n=(674, 674, 674)^\top$, $\mathbf P = (16,16,8)$ with normal 3d decomposition}
$\frac{674}{16}$:
\begin{itemize}
 \item $B = \left\lceil \frac{674}{16} \right\rceil = 43$
 \item $R = \left\lfloor \frac{674}{43} \right\rfloor = 15$
 \item $674 - R*B = 29$ 
\end{itemize}
\begin{equation*}
  \frac{674}{16} = \left[15\times 43, 1\times 29 \right]
\end{equation*}

$\frac{674}{8}$:
\begin{itemize}
 \item $B = \left\lceil \frac{674}{16} \right\rceil = 85$
 \item $R = \left\lfloor \frac{674}{43} \right\rfloor = 7$
 \item $674 - R*B = 79$
\end{itemize}
\begin{equation*}
  \frac{674}{16} = \left[7\times 85, 1\times 79 \right]
\end{equation*}

\subsubsection{Example: $\mathbf n=(674, 674, 674)^\top$, $\mathbf P = (16,16,8)$ with 3d decomposition suitable for 3d to 2d remap}

\begin{lstlisting}
/* n0/(p0*q0) x n1/(p1*q1) x n2 */
oblk[0] = PX(global_block_size)(n[0], PFFT_DEFAULT_BLOCK, p0*q0);
oblk[1] = PX(global_block_size)(n[1], PFFT_DEFAULT_BLOCK, p1*q1);
oblk[2] = n[2];

/* n0/p0 x n1/p1 x n2/(q0*q1) */
iblk[0] = oblk[0]*q0;
iblk[1] = oblk[1]*q1;
iblk[2] = PX(global_block_size)(n[2], PFFT_DEFAULT_BLOCK, q0*q1);
//PX(global_block_size)(blk1[2], PFFT_DEFAULT_BLOCK, q1);

/* n0/p0 x n1/(p1*q1) x n2/q0 */
mblk[0] = oblk[0]*q0;
mblk[1] = oblk[1];
mblk[2] = iblk[2]*q1;
\end{lstlisting}

Compute the three-dimensional block in the following way:
\begin{enumerate}
 \item compute blocks $B_0^\textrm{2d}, B_1^\textrm{2d}$ for 2d decomposition
 \item compute blocks for 3d decomposition via $B_0^\textrm{3d} = B_0^\textrm{2d} Q_0$, $B_1^\textrm{3d} = B_1^\textrm{2d} Q_1$ and the default block size $B_2^\textrm{3d} = \left\lceil\frac{n_2}{Q_0Q_1}\right\rceil$
\end{enumerate}


$P_0 = 16, P_1 = 16, Q_0=4, Q_1=2$

$\frac{n_0}{P_0Q_0} = \frac{674}{16\cdot 4}$:
\begin{itemize}
 \item $B_0^\textrm{2d} = \left\lceil \frac{674}{64} \right\rceil = 11$
 \item $B_0^\textrm{3d} = B_0^\textrm{2d}Q_0 = 44$
 \item $R_0^\textrm{3d} = \left\lfloor \frac{674}{44} \right\rfloor = 15$
 \item $674 - R_0^\textrm{3d}*B_0^\textrm{3d} = 15$
\end{itemize}
\begin{equation*}
  \frac{n_0}{P_0} = \frac{674}{16} = \left[15\times 44, 1\times 14 \right]
\end{equation*}
$\frac{n_1}{P_1Q_1} = \frac{674}{16\cdot 2}$:
\begin{itemize}
 \item $B_1^\textrm{2d} = \left\lceil \frac{674}{32} \right\rceil = 22$
 \item $B_1^\textrm{3d} = B_1^\textrm{2d}Q_1 = 44$
 \item $R_1^\textrm{3d} = \left\lfloor \frac{674}{44} \right\rfloor = 15$
 \item $674 - R_1^\textrm{3d}*B_1^\textrm{3d} = 15$
\end{itemize}
\begin{equation*}
  \frac{n_1}{P_1} = \frac{674}{16} = \left[15\times 44, 1\times 14 \right]
\end{equation*}

\begin{itemize}
 \item $B_2^\textrm{3d} = \left\lceil \frac{674}{8} \right\rceil = 85$
 \item $R_2^\textrm{3d} = \left\lfloor \frac{674}{85} \right\rfloor = 7$
 \item $674 - R_2^\textrm{3d}*B_2^\textrm{3d} = 79$
\end{itemize}
\begin{equation*}
  \frac{n_2}{Q_0Q_1} = \frac{674}{8} = \left[7\times 85, 1\times 79 \right]
\end{equation*}

%------------------------------------------------------------------------------
\section{Ghost cell support}
%------------------------------------------------------------------------------


%------------------------------------------------------------------------------
\section{Single and long double precision}
%------------------------------------------------------------------------------
